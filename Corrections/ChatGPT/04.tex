% Options for packages loaded elsewhere

%
\documentclass[
]{article}
\usepackage{amsmath,amssymb}
\usepackage{iftex}
\ifPDFTeX
  \usepackage[T1]{fontenc}
  \usepackage[utf8]{inputenc}
  \usepackage{textcomp} % provide euro and other symbols
\else % if luatex or xetex
  \usepackage{unicode-math} % this also loads fontspec
  \defaultfontfeatures{Scale=MatchLowercase}
  \defaultfontfeatures[\rmfamily]{Ligatures=TeX,Scale=1}
\fi
\usepackage{lmodern}
\ifPDFTeX\else
  % xetex/luatex font selection
\fi
% Use upquote if available, for straight quotes in verbatim environments
\IfFileExists{upquote.sty}{\usepackage{upquote}}{}
\IfFileExists{microtype.sty}{% use microtype if available
  \usepackage[]{microtype}
  \UseMicrotypeSet[protrusion]{basicmath} % disable protrusion for tt fonts
}{}
\makeatletter
\@ifundefined{KOMAClassName}{% if non-KOMA class
  \IfFileExists{parskip.sty}{%
    \usepackage{parskip}
  }{% else
    \setlength{\parindent}{0pt}
    \setlength{\parskip}{6pt plus 2pt minus 1pt}}
}{% if KOMA class
  \KOMAoptions{parskip=half}}
\makeatother
\usepackage{xcolor}
\setlength{\emergencystretch}{3em} % prevent overfull lines
\providecommand{\tightlist}{%
  \setlength{\itemsep}{0pt}\setlength{\parskip}{0pt}}
\setcounter{secnumdepth}{-\maxdimen} % remove section numbering
\ifLuaTeX
  \usepackage{selnolig}  % disable illegal ligatures
\fi
\usepackage{bookmark}
\IfFileExists{xurl.sty}{\usepackage{xurl}}{} % add URL line breaks if available
\urlstyle{same}
\hypersetup{
  hidelinks,
  pdfcreator={LaTeX via pandoc}}

\author{}
\date{}

\begin{document}

Here are the corrected answers, along with insights from the sources to
enhance your understanding:

\begin{enumerate}
\def\labelenumi{\arabic{enumi}.}
\item
  \textbf{Enhanced Mobile Broadband (eMBB), Ultra-Reliable Low-Latency
  Communications (URLLC), and Massive Machine-Type Communications
  (mMTC).}

  This is \textbf{correct}. The seminar states that these three service
  categories form the points of 5G's ``famous triangle.'' By 6G, the
  vision expands to a hexagon, incorporating additional pillars such as
  AI integration and integrated sensing and communication.
\item
  \textbf{Fully passive surfaces with reconfigurable elements controlled
  by a software-defined radio.}

  The initial answer is partially correct, but the sources highlight the
  diverse nature of RIS architectures. Currently, the seminar emphasizes
  \textbf{passive surfaces} with \textbf{reconfigurable elements} that
  are \textbf{controlled by software-defined radio}. This means that
  these surfaces reflect signals without amplifying them and can be
  adjusted to manipulate signal propagation. However, the speaker
  envisions future architectures with \textbf{active elements} capable
  of \textbf{amplification}, and even the possibility of \textbf{hybrid
  RIS} combining passive and active elements. This evolution suggests a
  trajectory towards more sophisticated and capable RIS in the future.
\item
  \textbf{Signal enhancement and coverage extension, particularly in
  shadowed areas, using signal reflections.}

  This is more accurate. The sources primarily focus on RIS's ability to
  \textbf{enhance signals} and \textbf{extend coverage}, especially in
  \textbf{shadowed areas} where signals struggle to penetrate. This is
  achieved through \textbf{strategically reflecting signals} to bypass
  obstacles and reach users more effectively. While \textbf{interference
  cancellation} is mentioned as a potential application, it is not the
  main focus of the seminar. The seminar demonstrates, through
  experiments, that RIS can significantly improve signal strength, with
  results showing increases of up to 10 dB in specific setups.
\item
  \textbf{RIS can be exploited by hackers to degrade signal quality or
  increase jamming, but it can also be used to enhance security by
  optimizing signal reflections to favor legitimate users and hinder
  eavesdroppers.}

  This accurately reflects the seminar's discussion on the security
  implications of RIS. While it can be a valuable tool for
  \textbf{enhancing security}, RIS is also vulnerable to
  \textbf{malicious exploitation}. Hackers could manipulate RIS to
  \textbf{degrade signal quality}, amplify \textbf{jamming attacks}, or
  even \textbf{eavesdrop} on communications. However, the seminar also
  explores how RIS can be used to \textbf{improve security} by
  \textbf{optimizing signal reflections} to favor legitimate users and
  create blind spots for eavesdroppers. This highlights the importance
  of secure RIS control and management to prevent its use for malicious
  purposes.
\item
  \textbf{Secure key generation for short-range, low-rate applications,
  potentially mitigating the future threat of quantum computers to
  current cryptographic techniques, particularly in IoT and other
  resource-constrained devices.}

  This is a more precise description. The seminar positions
  \textbf{noise-driven communication} as a potential method for
  \textbf{secure key generation}, but it emphasizes its suitability for
  \textbf{short-range, low-rate applications} like those commonly found
  in IoT. It is presented as a way to address the looming threat of
  \textbf{quantum computers}, which could potentially break the
  cryptographic techniques widely used today. However, the seminar does
  not explicitly classify it as ``quantum-resistant communication,''
  focusing instead on its practicality for resource-constrained devices
  where implementing robust cryptographic measures might be challenging.
  The speaker suggests that noise-based systems could be used for key
  generation within data centers or even between components within a
  computer, highlighting the potential for localized, secure
  communication.
\end{enumerate}

\end{document}
