% Options for packages loaded elsewhere

%
\documentclass[
]{article}
\usepackage{amsmath,amssymb}
\usepackage{iftex}
\ifPDFTeX
  \usepackage[T1]{fontenc}
  \usepackage[utf8]{inputenc}
  \usepackage{textcomp} % provide euro and other symbols
\else % if luatex or xetex
  \usepackage{unicode-math} % this also loads fontspec
  \defaultfontfeatures{Scale=MatchLowercase}
  \defaultfontfeatures[\rmfamily]{Ligatures=TeX,Scale=1}
\fi
\usepackage{lmodern}
\ifPDFTeX\else
  % xetex/luatex font selection
\fi
% Use upquote if available, for straight quotes in verbatim environments
\IfFileExists{upquote.sty}{\usepackage{upquote}}{}
\IfFileExists{microtype.sty}{% use microtype if available
  \usepackage[]{microtype}
  \UseMicrotypeSet[protrusion]{basicmath} % disable protrusion for tt fonts
}{}
\makeatletter
\@ifundefined{KOMAClassName}{% if non-KOMA class
  \IfFileExists{parskip.sty}{%
    \usepackage{parskip}
  }{% else
    \setlength{\parindent}{0pt}
    \setlength{\parskip}{6pt plus 2pt minus 1pt}}
}{% if KOMA class
  \KOMAoptions{parskip=half}}
\makeatother
\usepackage{xcolor}
\setlength{\emergencystretch}{3em} % prevent overfull lines
\providecommand{\tightlist}{%
  \setlength{\itemsep}{0pt}\setlength{\parskip}{0pt}}
\setcounter{secnumdepth}{-\maxdimen} % remove section numbering
\ifLuaTeX
  \usepackage{selnolig}  % disable illegal ligatures
\fi
\usepackage{bookmark}
\IfFileExists{xurl.sty}{\usepackage{xurl}}{} % add URL line breaks if available
\urlstyle{same}
\hypersetup{
  hidelinks,
  pdfcreator={LaTeX via pandoc}}

\author{}
\date{}

\begin{document}

Here are the corrected answers, along with insights from the sources to
enhance your understanding:

\begin{enumerate}
\def\labelenumi{\arabic{enumi}.}
\item
  \textbf{Enhanced Mobile Broadband (eMBB), Ultra-Reliable Low Latency
  Communications (URLLC), and Massive Machine-Type Communications
  (mMTC).}

  This is \textbf{correct}. The seminar mentions that these three
  service categories make up the points of 5G's ``famous triangle,''
  representing the key focus areas of 5G technology. Looking ahead to
  6G, the vision expands to a hexagon, with additional pillars like AI
  integration and integrated sensing and communication added to the
  original three. This expansion reflects the changing landscape of
  wireless communication needs as we move toward 6G and beyond.
\item
  \textbf{Reconfigurable Intelligent Surfaces (RIS)}

  The provided answer is close, but needs a slight adjustment. The
  correct terminology used in the sources is \textbf{Reconfigurable
  Intelligent Surfaces (RIS)}, not IRIS. This term more accurately
  reflects the nature of these surfaces -- they are designed to be
  reconfigurable, allowing for dynamic adjustments to their properties
  to control and optimize signal propagation.
\item
  \textbf{Enhancing wireless coverage and providing secure communication
  through signal manipulation.}

  This is a more accurate representation based on the information
  provided in the sources. The seminar heavily emphasizes the ability of
  RIS to \textbf{enhance wireless coverage}, particularly in areas with
  weak signal strength or where obstacles impede signal propagation.
  This is achieved by strategically \textbf{reflecting signals} off the
  RIS to bypass obstacles and reach users more effectively.

  The sources also discuss the potential of RIS to \textbf{improve
  security in wireless communication}. By controlling the reflection
  properties of RIS, it is possible to \textbf{favor legitimate users
  and hinder eavesdroppers}, making it more difficult for unauthorized
  parties to intercept sensitive information.

  While \textbf{improving energy efficiency} is a significant
  consideration in wireless communication systems, the seminar does not
  explicitly highlight this as a primary benefit of using RIS.
  Therefore, it's best to focus on the aspects that are directly
  discussed in the sources.
\item
  \textbf{By strategically reflecting and focusing signals, RIS can
  enhance coverage, mitigate interference, and improve security in
  wireless communication systems.}

  This statement provides a more precise explanation of how RIS
  functions to achieve the benefits discussed in the sources. RIS, as
  the seminar explains, works by \textbf{strategically reflecting and
  focusing signals}. This capability is the foundation for its various
  applications:

  \begin{itemize}
  
  \item
    \textbf{Enhancing Coverage:} By reflecting signals around obstacles
    and directing them towards users in shadowed areas, RIS can
    significantly improve coverage.
  \item
    \textbf{Mitigating Interference:} RIS can also be used to reflect
    signals away from areas where they might cause interference, thereby
    improving overall signal quality.
  \item
    \textbf{Improving Security:} By strategically focusing signals
    toward legitimate users and creating blind spots for potential
    eavesdroppers, RIS can contribute to enhancing the security of
    wireless communication.
  \end{itemize}

  The seminar focuses on the practical applications of RIS, emphasizing
  the role of signal manipulation in achieving desired outcomes.
\item
  \textbf{Noise-driven communication could be used for secure
  communication in short-range, low-rate applications and as a potential
  solution to the threat of quantum computers, especially for
  resource-constrained devices like those in IoT.}

  This statement accurately reflects the seminar's discussion of
  noise-driven communication and its potential applications. The seminar
  highlights its suitability for \textbf{secure communication in
  short-range, low-rate applications}, which are commonly found in the
  Internet of Things (IoT).

  The seminar also emphasizes the looming threat of \textbf{quantum
  computers}, which could potentially break the widely used
  cryptographic techniques that protect our digital world today.
  Noise-driven communication is presented as a possible solution to this
  challenge, especially for \textbf{resource-constrained devices} where
  implementing complex cryptographic measures might be challenging.

  While the seminar discusses the use of noise for secure key
  generation, it does not explicitly claim that noise-driven
  communication offers ``quantum communication'' capabilities. It is
  essential to stay within the boundaries of what the sources explicitly
  state.
\end{enumerate}

\end{document}
