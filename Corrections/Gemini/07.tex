% Options for packages loaded elsewhere

%
\documentclass[
]{article}
\usepackage{amsmath,amssymb}
\usepackage{iftex}
\ifPDFTeX
  \usepackage[T1]{fontenc}
  \usepackage[utf8]{inputenc}
  \usepackage{textcomp} % provide euro and other symbols
\else % if luatex or xetex
  \usepackage{unicode-math} % this also loads fontspec
  \defaultfontfeatures{Scale=MatchLowercase}
  \defaultfontfeatures[\rmfamily]{Ligatures=TeX,Scale=1}
\fi
\usepackage{lmodern}
\ifPDFTeX\else
  % xetex/luatex font selection
\fi
% Use upquote if available, for straight quotes in verbatim environments
\IfFileExists{upquote.sty}{\usepackage{upquote}}{}
\IfFileExists{microtype.sty}{% use microtype if available
  \usepackage[]{microtype}
  \UseMicrotypeSet[protrusion]{basicmath} % disable protrusion for tt fonts
}{}
\makeatletter
\@ifundefined{KOMAClassName}{% if non-KOMA class
  \IfFileExists{parskip.sty}{%
    \usepackage{parskip}
  }{% else
    \setlength{\parindent}{0pt}
    \setlength{\parskip}{6pt plus 2pt minus 1pt}}
}{% if KOMA class
  \KOMAoptions{parskip=half}}
\makeatother
\usepackage{xcolor}
\setlength{\emergencystretch}{3em} % prevent overfull lines
\providecommand{\tightlist}{%
  \setlength{\itemsep}{0pt}\setlength{\parskip}{0pt}}
\setcounter{secnumdepth}{-\maxdimen} % remove section numbering
\ifLuaTeX
  \usepackage{selnolig}  % disable illegal ligatures
\fi
\usepackage{bookmark}
\IfFileExists{xurl.sty}{\usepackage{xurl}}{} % add URL line breaks if available
\urlstyle{same}
\hypersetup{
  hidelinks,
  pdfcreator={LaTeX via pandoc}}

\author{}
\date{}

\begin{document}

{Correcting and Expanding on Statements about Scientific
Machine
Learning}\label{correcting-and-expanding-on-statements-about-scientific-machine-learning}

Let's examine and refine the statements about Physics-Informed Neural
Networks (PINNs), Reduced Order Models (ROMs), Operator Inference, and
Quantum Computing in scientific machine learning.

\begin{itemize}
\item
  \textbf{1- PINNs are data-efficient and physics-informed but
  computationally expensive.}

  This statement is \textbf{correct}. The sources highlight that PINNs
  are designed to incorporate physical laws, leading to improved
  accuracy and reduced data requirements. However, training PINNs,
  particularly for complex systems, can be computationally demanding.
  The seminar specifically mentions the challenges associated with
  hyperbolic PDEs, which govern many important physical phenomena.
\item
  \textbf{2- The team transitioned to ROMs for computational
  efficiency.}

  This statement is also \textbf{correct}. The sources explicitly state
  that the shift from PINNs to ROMs was motivated by the need for
  improved computational efficiency and scalability. ROMs are
  particularly advantageous for applications involving multiple queries
  or simulations, where the computational cost of PINNs becomes
  prohibitive.
\item
  \textbf{3- ROMs accelerate simulations but can be less accurate.}

  This statement is \textbf{correct}. ROMs achieve acceleration by
  approximating the system's dynamics with reduced complexity,
  effectively reducing the dimensionality of the problem. While this
  simplification enables faster simulations, it inevitably introduces a
  degree of approximation, potentially leading to reduced accuracy,
  especially for systems with complex or nonlinear behavior. The
  trade-off between speed and accuracy is a fundamental aspect of using
  ROMs.
\item
  \textbf{4- Operator inference stands out by learning underlying
  physical laws.}

  This statement requires some \textbf{clarification}. While Operator
  Inference does leverage knowledge of the underlying physical system,
  it doesn't directly ``learn'' physical laws in the same way as PINNs.
  Instead, Operator Inference focuses on learning the \emph{operators}
  that govern the system's dynamics, often through a data-driven
  approach. This operator-focused learning can lead to improved
  generalization and extrapolation capabilities. The seminar emphasizes
  that Operator Inference is favored for its computational efficiency,
  relying on simpler regression methods and straightforward
  regularization compared to DeepONets and equivariant neural networks.
\item
  \textbf{5- Quantum computing can accelerate scientific machine
  learning, including greenhouse gas emission estimation.}

  This statement is \textbf{correct}, and the seminar presents a
  specific roadmap for applying quantum computing to greenhouse gas
  emission estimation. The approach uses a hybrid classical-quantum
  model where quantum algorithms, particularly quantum regression, are
  integrated into the Operator Inference workflow. The aim is to
  leverage quantum computing's potential for handling complex
  calculations to enhance the speed and accuracy of emission
  estimations. The sources, however, acknowledge that the practical
  application of quantum computing in this domain is still in its early
  stages and relies on the continued development of quantum hardware and
  algorithms. The seminar stresses the importance of demonstrating
  ``utility'' for quantum computing in real-world applications, and the
  proposed roadmap for greenhouse gas emission estimation serves as an
  example of this effort.
\end{itemize}

\end{document}
