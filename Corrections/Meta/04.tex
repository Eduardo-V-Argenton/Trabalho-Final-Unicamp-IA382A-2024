% Options for packages loaded elsewhere

%
\documentclass[
]{article}
\usepackage{amsmath,amssymb}
\usepackage{iftex}
\ifPDFTeX
  \usepackage[T1]{fontenc}
  \usepackage[utf8]{inputenc}
  \usepackage{textcomp} % provide euro and other symbols
\else % if luatex or xetex
  \usepackage{unicode-math} % this also loads fontspec
  \defaultfontfeatures{Scale=MatchLowercase}
  \defaultfontfeatures[\rmfamily]{Ligatures=TeX,Scale=1}
\fi
\usepackage{lmodern}
\ifPDFTeX\else
  % xetex/luatex font selection
\fi
% Use upquote if available, for straight quotes in verbatim environments
\IfFileExists{upquote.sty}{\usepackage{upquote}}{}
\IfFileExists{microtype.sty}{% use microtype if available
  \usepackage[]{microtype}
  \UseMicrotypeSet[protrusion]{basicmath} % disable protrusion for tt fonts
}{}
\makeatletter
\@ifundefined{KOMAClassName}{% if non-KOMA class
  \IfFileExists{parskip.sty}{%
    \usepackage{parskip}
  }{% else
    \setlength{\parindent}{0pt}
    \setlength{\parskip}{6pt plus 2pt minus 1pt}}
}{% if KOMA class
  \KOMAoptions{parskip=half}}
\makeatother
\usepackage{xcolor}
\setlength{\emergencystretch}{3em} % prevent overfull lines
\providecommand{\tightlist}{%
  \setlength{\itemsep}{0pt}\setlength{\parskip}{0pt}}
\setcounter{secnumdepth}{-\maxdimen} % remove section numbering
\ifLuaTeX
  \usepackage{selnolig}  % disable illegal ligatures
\fi
\usepackage{bookmark}
\IfFileExists{xurl.sty}{\usepackage{xurl}}{} % add URL line breaks if available
\urlstyle{same}
\hypersetup{
  hidelinks,
  pdfcreator={LaTeX via pandoc}}

\author{}
\date{}

\begin{document}

Here are the corrected answers, along with insights from the sources to
enhance your understanding:

\begin{enumerate}
\def\labelenumi{\arabic{enumi}.}
\item
  \textbf{Enhanced Mobile Broadband (eMBB), Ultra-Reliable Low-Latency
  Communications (URLLC), and Massive Machine-Type Communications
  (mMTC).}

  This is \textbf{correct}. The seminar highlights that these three
  service categories constitute the foundation of 5G's ``famous
  triangle,'' symbolizing the key areas of focus for 5G technology. As
  we progress toward 6G, the vision expands from a triangle to a
  hexagon, incorporating new pillars such as AI integration and
  integrated sensing and communication. This expansion underscores the
  evolving landscape of wireless communication needs as we move toward
  6G and beyond.
\item
  \textbf{The ultimate RIS architecture envisions a system that
  integrates passive and active elements, potentially incorporating
  amplification and signal processing capabilities, enabling more
  sophisticated control over signal manipulation.}

  While a fully programmable and autonomous system is a likely goal for
  future RIS development, the sources suggest a more nuanced picture.
  The seminar emphasizes the current focus on \textbf{passive RIS},
  which primarily reflect signals without the need for amplification or
  complex signal processing. However, the speaker envisions future RIS
  architectures incorporating both \textbf{passive and active elements}.
  These \textbf{active elements} would enable functionalities such as
  \textbf{signal amplification}, providing greater control over signal
  strength and direction. The seminar also mentions \textbf{hybrid RIS},
  which combine passive and active components, offering a balance
  between functionality and complexity.

  The idea of a fully autonomous system adapting to changing
  environmental conditions is not explicitly discussed in the seminar.
  While this level of autonomy could be a future aspiration, the sources
  primarily focus on the evolution of RIS architectures toward more
  sophisticated signal manipulation capabilities.
\item
  \textbf{The most promising applications for RIS, according to the
  seminar, center around enhancing wireless coverage, particularly in
  shadowed areas and indoor environments, and improving security by
  manipulating signal reflections to favor legitimate users and hinder
  eavesdroppers.}

  The seminar primarily focuses on the potential of RIS to
  \textbf{enhance wireless coverage}, especially in \textbf{shadowed
  areas} where obstacles obstruct signal propagation, and in
  \textbf{indoor environments} where signals struggle to penetrate walls
  and other barriers. The sources highlight the ability of RIS to
  reflect signals strategically, directing them around obstacles and
  towards users, thereby improving signal strength and coverage.

  The seminar also underscores the role of RIS in \textbf{enhancing
  security}. By manipulating signal reflections, RIS can be configured
  to \textbf{favor legitimate users} while simultaneously
  \textbf{creating blind spots for potential eavesdroppers}. This
  selective signal manipulation can make it significantly more difficult
  for unauthorized parties to intercept sensitive information.

  While \textbf{6G and terahertz communications} are likely areas where
  RIS could play a role, the seminar does not explicitly focus on these
  technologies. Similarly, \textbf{wireless sensing and tracking} are
  not mentioned as primary applications for RIS in the seminar. The
  seminar's emphasis lies squarely on the potential of RIS to
  revolutionize wireless coverage and security through intelligent
  signal manipulation.
\item
  \textbf{RIS can enhance security by optimizing signal reflections to
  create favorable conditions for legitimate users while hindering
  eavesdroppers.}

  The seminar emphasizes the potential of RIS to \textbf{improve
  security} in wireless communication systems by intelligently
  controlling signal reflections. The sources highlight the ability of
  RIS to \textbf{create favorable conditions for legitimate users},
  ensuring strong signal reception and reliable communication. At the
  same time, RIS can be configured to \textbf{hinder eavesdroppers} by
  directing signals away from them, effectively creating blind spots
  that make it difficult to intercept sensitive information.

  While the seminar does not explicitly mention RIS acting as a ``smart
  jammer,'' the principle of using RIS to disrupt unauthorized access is
  similar. By strategically manipulating signal reflections, RIS can
  create an environment that favors authorized communication while
  hindering unauthorized access, thereby enhancing security.
\item
  \textbf{Potential applications of noise-driven communication in a
  world with quantum computers include secure key generation for
  short-range, low-rate applications, particularly in IoT and other
  resource-constrained environments, and potentially mitigating the
  threat posed by quantum computers to current cryptographic
  techniques.}

  The seminar highlights the potential of \textbf{noise-driven
  communication}, especially in a future where \textbf{quantum
  computers} pose a significant threat to current cryptographic
  techniques. The sources emphasize its suitability for \textbf{secure
  communication in short-range, low-rate applications}, which are
  characteristic of many \textbf{IoT devices and other
  resource-constrained environments}.

  Noise-driven communication, according to the seminar, can be used for
  \textbf{secure key generation}, enabling secure communication without
  relying on traditional cryptographic methods that might be vulnerable
  to quantum computers. While the seminar does not explicitly mention
  \textbf{random number generation}, this could be a potential
  application of noise-driven communication, given the inherent
  randomness of noise.

  It is important to note that the seminar \textbf{does not explicitly
  mention noise-driven communication as a form of ``quantum
  communication.''} While the technology might offer some security
  advantages in a post-quantum world, it does not leverage the
  principles of quantum mechanics for communication, as quantum key
  distribution (QKD) does. The seminar focuses on the practical aspects
  of using noise for secure communication, particularly its
  applicability to resource-constrained devices and its potential to
  address future security challenges posed by quantum computers.
\end{enumerate}

\end{document}
